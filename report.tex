\documentclass{article}

\usepackage[utf8]{inputenc}
%\usepackage{xunicode}
\usepackage[T1]{fontenc}
%\usepackage{fontspec}
\usepackage{geometry}
\usepackage[english]{babel}
%\usepackage{polyglossia}
%\setmainlanguage{english}
\usepackage{amsmath}
\usepackage{amssymb}
%\usepackage{unicode-math}
\usepackage{hyperref}
\usepackage{enumitem}
\setlist{nosep,leftmargin=*}
\frenchspacing

\newcommand\setR{\mathbb{R}}
\newcommand\setZ{\mathbb{Z}}
\newcommand\setN{\mathbb{N}}

\title{Array expansion in function arguments and return values}
\author{Quentin \textsc{Corradi}}
\setlength{\parindent}{0pt}

% TODO: biblio (array contraction)

\begin{document}

\maketitle

\section{Introduction}

Arrays are an abstraction for handling memory using integers. It abstracts
contiguous memory into a partial map from integers to memory cells. This
abstraction could also be conceived as a partial map from integers to variables.
This means any array \(t\) of size \(N\) could be replaced by \(N\) variables
\(t_0, \dots, t_{N - 1}\) and a pair of functions
\((\mathit{get}, \mathit{set})\) such that \(\mathit{get} : k \mapsto t_k\) and
\(\mathit{set}\;k\;x\) sets \(t_k\) to \(x\).

\smallskip

The idea of array expansion is that the above interpretation of arrays can be
used litterally as it doesn't need contiguity of variables in memory or any
hardware support for integer indexing while still allowing all common operations
on arrays: An array whose size is known can be replaced during compilation by as
many variables as necessary, simplifying access to some kind of indicies to
simple variable accesses.

\smallskip

Jasmin is a compiler targetted at ???.
 Jasmin est un compilateur utilisé pour la crypto, il est associé d'une preuve
 de correction et de préservation de constant-time.
The Jasmin language is close to assembly, the main differences lies in:
\begin{itemize}
\item Jasmin abstracts numerical operations common in cryptography
\item Jasmin provides variables and performs variable allocation in registers
\item Jasmin supports arrays
\item Jasmin provides common control flow statements
\item Jasmin provides functions with support for C calling convention if needed
\end{itemize}\smallskip

In Jasmin there are two kind of variables, the stack allocated ones and the
register allocated ones. The user has to tell the compiler the kind of every
variables. This also holds true for arrays. However on most processor on which
people run ??? software there is no hardware support for indexing registers with
a value not encoded in the instruction. So some compilation step eliminates
these arrays.

\smallskip

In this article we did not implement most of the array expansion the Jasmin
compiler already performs. Instead we extended this transformation for some
previously unsupported cases, namely expanding register arrays in function
arguments and return values. 

\medskip

This article is organised as follows:
In Section \ref{sec:syntax} the relevant parts of Jasmin syntax trees are
presented then the syntactic side of array expansion is explained more in-depth
for our intended use in Jasmin.
In Section \ref{sec:semantics} the relevant parts of Jasmin semantics are
presented then the semantic side of array expansion is explored.
% TODO: In section \ref{sec:???}

 Présentation des modifications de la sémantique au niveau des fonctions ainsi
 que des varmap, avec les propriétés importantes et l'élimination des types
 dépendants pour justifier les choix. Et de la transformation effectuée.

 Présentation du travail restant dans le stage ainsi que potentiellement après
 le stage: FIXER LES PREUVES


\section{Array expansion compiler pass}\label{sec:syntax}

\subsection{Introduction to relevant parts of Jasmin syntax}

In this section, the relevant parts of Jasmin programs are described. The
irrelevant parts, if mentioned, will not be expanded upon.

\begin{figure}
\obeylines\obeyspaces\ttfamily%
type ty =
| Word of wsize
| Arr  of wsize \(\times\;\setZ\)

type v\_kind =
| Stack of pointer
| Reg   of reg\_kind \(\times\) pointer

type var = \{
~ v\_id   : uid;
~ v\_kind : v\_kind;
~ v\_ty   : ty;
\}

type expr =
| Pconst of \(\setZ\)
| Parr\_init of \(\setZ\)
| Pvar   of var
| Pget   of arr\_access \(\times\) wsize \(\times\) var \(\times\) expr
| Psub   of arr\_access \(\times\) wsize \(\times\;\setZ\;\times\) var \(\times\) expr
| Pload  of wsize \(\times\) var \(\times\) expr

type lval =
| Lnone
| Lvar  of var
| Lmem  of wsize \(\times\) var \(\times\) expr
| Laset of arr\_access \(\times\) wsize \(\times\) var \(\times\) expr
| Lasub of arr\_access \(\times\) wsize \(\times\;\setZ\;\times\) var \(\times\) expr

type instr =
| Cassgn of lval \(\times\) expr
| Ccall  of lval list \(\times\) uid \(\times\) expr list

type call\_conv =
| Export
| Subroutine

type func = \{
~ f\_cc   : call\_conv;
~ f\_name : uid;
~ f\_args : var list;
~ f\_body : instr list;
~ f\_ret  : var list;
\}

type prog = func list
\normalfont%
\caption{Simplified version of the relevant parts of Jasmin programs AST.}
\end{figure}

A Jasmin progam is a list of functions among other things. A function is made
among other things of its name, a unique identifier; its parameters, a list of
variables; its body, a list of statements; its returned variables, a list of
variables; and its kind. The kind of a function can be exported to the outside,
in which case the function must follow some calling convention, or internal.

\smallskip

A statement can be an assignment or a function call.
Other statements include assembly instructions and control flow statements,
however they are not relavant to us.

An assignment is made of a left member and an expression. A function call is
made of the called function identifier, a list of left members matching the
function returned variables list on length, and a list of expression matching
the function parameters list on length.

Function arguments and return variables, left members and expressions refer to
variables. A variable is made of an identifier, a type and a kind. In the AST a
variable doesn't have to be declared in order to be used. Across a function
body, parameters and returned variables, all variables featuring the same
identifier are considered to be the same. A variable kind is either stack or
register. It indicates where the variable is stored.

\smallskip

A type can be a word of some size (a power of 2 bytes) or an array of some
length over words of some size. Arrays are indexed by two means, the first is
by considering the array as cells of some size, the other is by considering it
as a sequence of bytes (independently of the size of the data being accessed at
the index).

Arrays can be built out of other arrays by taking a slice. A slice is a
contiguous section of some length at an offset of the begining of an array.

\smallskip

A left members is either a variable, a cell of an array variable, a slice of an
array variable, a memory cell or the discard.

An array cell is indexed by an expression, its size is a word size. An array
slice length and offset are expressions, the cell size is a word size. A memory
cell is an array cell but the base array is the process memory instead of a
variable. The discard left member does not store the value it is assigned.

An expression can be a constant word, an empty array, a variable, a cell of an
array variable, a slice of an array variable or a memory cell. Operations are
also expressions made of other expressions.

\subsection{Array expansion in Jasmin compiler}

A variable of type array and kind register means an array of register, but in
most common processor on which ??? programs are run there is no hardware support
for register indexing outside of the processor instruction opcode. Therefore we
need to eliminate them in some compilation step; this is what the array
expansion step does in the Jasmin compiler. Other kind of arrays are not
expanded --even though we could expand them-- because there is hardware support
for the operation performed on them and so we consider users know what they want.

\medskip

We want to replace all store and load operations on an array. There are two way
to index on array as described in the previous section. The accesses where the
index is an offset in bytes from the start of the array would require
concatenating parts of registers together on loads or assigning portions of a
value into portions of registers on stores. While this is completely possible we
do not want to provide it so a program with this access scheme used on a
register array will fail to compile.
The accesses we have to deal with are of three kind: single cell indexing,
array slicing and whole array access.

\smallskip

Assume we want to expand an array into variables. Single cell indexing can be
resolved to a single variable when the index is a constant. Otherwise we do not
support it. The exception is when the index is a for loop variable, in which
case at this compilation step it is a constant because loop unrolling has
already been performed.
Currently only this part is implemented in Jasmin.

Whole array access can be resolved either by introducing a real array on which
the operation is performed, or when the operation can be unrolled by unrolling
it and using the method for single cell indexing. We won't introduce
intermediate arrays because we do not want to incur hidden costs. If the user
wants to introduce an intermediate array they can and the resulting operation
can be unrolled. 

Finally array slicing would only be supported when the parameters of the slice
are constant. In this case the slice is a new array and the variable it is
replaced by is a slice with the same parameters of the variables by which the
original array would be replaced by.

\smallskip

Another kind of expansion not falling under the above cases is when a function
takes a register array as argument. In this case we instead take as many
register variables as necessary instead of the array.
That transformation is only wanted when the function is not exported to the
other programs because we do not want to make such changes unbeknown to the
user. Jasmin could either warn the user or as we implemented it, fail to
compile the program.
Only array expansion for internal functions will be implemented in this article
even though our changes enables the other unimplemented expansion above.


\section{Semantics aspect of array expansion}\label{sec:semantics}

\subsection{Relevant parts of Jasmin semantics}

In this section, the relevant parts of Jasmin semantics are described.

\begin{figure}
\obeylines\obeyspaces\ttfamily%
Variant value :=
~ | Varr  n : array n \(\rightarrow\) value
~ | Vword s : word  s \(\rightarrow\) value.

Definition sem\_t t :=
~ match t with
~ | Arr  n \(\Rightarrow\) array n
~ | Word s \(\Rightarrow\) word  s
~ end.
\normalfont%
\caption{Simplified value types.}\label{fig:val}
\end{figure}

Jasmin values can be words and pratially defined arrays of words. Variables value
are stored using the dependant type \texttt{sem\_t} as defined in Figure
\ref{fig:val} in order to keep the invariant
\texttt{m.[x] = Ok v \(\rightarrow\) type\_of\_val v = v\_ty x}.

\begin{figure}
\obeylines\obeyspaces\ttfamily%
Definition to\_arr  n v := if v is Varr  n t then Ok t else Error.

Definition to\_word s v := if v is Vword s w then Ok w else Error.

Definition to\_val t : sem\_t t \(\rightarrow\) value :=
~ match t return sem\_t t \(\rightarrow\) value with
~ | Arr  n \(\Rightarrow\) Varr  n
~ | Word s \(\Rightarrow\) Vword s
~ end.
\normalfont%
\caption{Value conversion between \texttt{value} and \texttt{sem\_t}.}
\end{figure}

\begin{figure}
\obeylines\obeyspaces\ttfamily%
Fixpoint sem\_pexpr s e :=
~ match e with
~ | Pconst z \(\Rightarrow\) Ok (Vword z)
~ | Parr\_init n \(\Rightarrow\) Ok (Varr n (empty n))
~ | Pvar v \(\Rightarrow\) s.[v]
~ | Pget aa ws x e \(\Rightarrow\)
~   Let Varr n t := s.[x] in
~   Let i := sem\_pexpr s e >{}>= to\_word in
~   Let w := get aa ws t i in
~   Ok (Vword w)
~ | Psub aa ws len x e \(\Rightarrow\)
~   Let Varr n t := s.[x] in
~   Let i := sem\_pexpr s e >{}>= to\_word in
~   Let t' := get\_sub aa ws len t i in
~   Ok (Varr len t')
~ end.

Definition write\_lval l v s :=
~ match l with
~ | Lnone \(\Rightarrow\) Ok s
~ | Lvar x \(\Rightarrow\) s.[x \(\leftarrow\) v]
~ | Laset aa ws x i \(\Rightarrow\)
~   Let Varr n t := s.[x] in
~   Let i := sem\_pexpr s i >{}>= to\_word in
~   Let v := to\_word ws v in
~   Let t := set t aa i v in
~   s.[x \(\leftarrow\) to\_val (Arr n) t]
~ | Lasub aa ws len x i \(\Rightarrow\)
~   Let Varr n t := s.[x] in
~   Let i := sem\_pexpr s i >{}>= to\_word in
~   Let t' := to\_arr (ws * len) v in 
~   Let t := set\_sub n aa ws len t i t' in
~   s.[x \(\leftarrow\) to\_val (Arr n) t]
~ end.
\normalfont%
\caption{On the top: function evaluating expression; %
On the bottom: function affecting left members.}
\end{figure}

% TODO: présenter les règles d'evaluation des expressions et les règles des
% left-members

\begin{figure}
\ttfamily
sem\_call P :
\begin{center}
\begin{tabular}{c}
get\_func P fn = Some f \\
fold2 (\(\lambda\)x v, s.[x \(\leftarrow\) v]) (f\_args f) vargs empty = Ok s1 \\
sem P s1 (f\_body f) s2 \\
mapM (\(\lambda\)x, s2.[x]) (f\_ret f) = Ok vres
\\\hline
sem\_call P fn vargs vres
\end{tabular}
\end{center}

sem P :
\begin{center}
\begin{tabular}{c}
\\\hline
sem P s [::] s
\end{tabular} Eskip\quad
\begin{tabular}{c}
sem\_i P s1 i s2\quad
sem P s2 l s3
\\\hline
sem P s1 (i::l) s3
\end{tabular} Eseq
\end{center}

sem\_i P :
\begin{center}
\begin{tabular}{c}
sem\_pexpr s1 e = Ok v\quad
write\_lval x v s1 = Ok s2
\\\hline
sem\_i P s1 (Cassgn x e) s2
\end{tabular} Eassgn

~

\begin{tabular}{c}
sem\_pexprs s1 args = Ok vargs \\
sem\_call P f vargs vres \\
write\_lvals s1 xs vres = Ok s2
\\\hline
sem\_i P s1 (Ccall xs f args) s2.
\end{tabular} \raisebox{-\baselineskip}{Ecall}
\end{center}
\normalfont%
\caption{Inductive big step semantics rules.}
\end{figure}
% TODO: Présenter les règles de la bigstepsem


% Présentation des sémantiques actuelle des valeurs (array never undef) et des
% sémantiques (sur des exemples puis pour) des appels de fonctions

\subsection{Semantic implications of array expansion}

% Présenter la problématique introduite par cette transformation qui la rendrait
% improuvable puis les pistes de modifications.


\section{}
% Présentation des modifications de la sémantique au niveau des fonctions ainsi
% que des varmap, avec les propriétés importantes et l'élimination des types
% dépendants pour justifier les choix. Et de la transformation effectuée.

\section{Conclusion}
% Présentation du travail restant dans le stage ainsi que potentiellement après
% le stage: FIXER LES PREUVES

\end{document}
