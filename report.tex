\documentclass{article}

%\usepackage{xunicode}
\usepackage[T1]{fontenc}
%\usepackage{fontspec}
\usepackage{geometry}
\usepackage[english]{babel}
%\usepackage{polyglossia}
%\setmainlanguage{english}
\usepackage{amsmath}
\usepackage{amssymb}
%\usepackage{unicode-math}
\usepackage[hidelinks]{hyperref}
\usepackage{enumitem}
\setlist{nosep,leftmargin=*}
\frenchspacing

\newcommand\setR{\mathbb{R}}
\newcommand\setZ{\mathbb{Z}}
\newcommand\setN{\mathbb{N}}

\title{Improving array expansion in Jasmin compiler}
\author{Quentin \textsc{Corradi}}
\setlength{\parindent}{0pt}

\begin{document}

\maketitle

\section{Introduction}

Jasmin\cite{10.1145/3133956.3134078, 10.1145/3319535.3363211, 9152665} is a
compiler and a language designed for writing high-assurance and
high-speed cryptography primitives. The semantics is formally defined to allow
rigorous reasoning about program behaviors.
The Jasmin language is close to assembly, the main differences lie in:
\begin{itemize}
\item Jasmin abstracts numerical operations common in cryptography
\item Jasmin provides variables and performs variable allocation in registers
\item Jasmin supports arrays
\item Jasmin provides common control flow statements
\item Jasmin provides functions with support for C calling convention if needed
\item Jasmin presents assembly instruction with a functional style to avoid
hidden side effects
\end{itemize}

In Jasmin there are two kind of variables, the stack allocated ones and the
register allocated ones. The user has to tell the compiler the kind of every
variables. This also holds true for arrays. However on most processors on which
people run cryptographic primitives and software, there is no hardware support
for indexing registers with a value, all registers need to be known at
compilation. In Jasmin a compilation pass does it.

\medskip

In most programming languages, arrays are an abstraction for handling memory
using integers. It abstracts contiguous memory into a partial map from integers
to memory cells in order to hide arithmetic on pointers. This abstraction could
also be thought as a partial map from integers to variables, that is any array
\(t\) of size \(N\) could be replaced by \(N\) variables
\(t_0, \dots, t_{N - 1}\) and a pair of functions
\((\mathit{get}, \mathit{set})\) such that \(\mathit{get} : k \mapsto t_k\) and
\(\mathit{set}\;k\;x\) sets \(t_k\) to \(x\).

This way of thinking about arrays allows us to eliminate Jasmin register arrays
by replacing them by variables. This compilation pass is called array expansion
and is not a standard compilation pass. In the following array expansion will
also refer to the action of eliminating arrays by replacing them with the
appropriate amount of variables.

\medskip

In this article we try to extend the array expansion pass the Jasmin compiler
already performs. In particular we implemeted array expansion in function
signature and function calls, and we studied how more array expansion could be
performed in particular for slice assignments informed by how the expansion of
whole arrays in function call works.

\medskip

This article is organised as follows:
In Section \ref{sec:syntax} the relevant parts of Jasmin syntax trees are
presented then the syntactic side of array expansion is explained more in-depth
for our intended use in Jasmin.
In Section \ref{sec:semantics} the relevant parts of Jasmin semantics are
presented then the semantics side of array expansion is explored.
In Section \ref{sec:perch} we describe and justify a change we made to the core
of Jasmin semantics and the consequent changes we had to make in order to
accomodate it and fix the correctness theorems.

\begin{figure}[t]
\obeylines\obeyspaces\ttfamily%
\begin{minipage}{0.45\textwidth}
type wsize = \(\{ 2^i \mid i \in \setN \}\)

type ty =
| Word of wsize
| Arr  of wsize \(\times\;\setN\)

type v\_kind =
| Stack
| Reg

type var = \{
~ v\_id   : uid;
~ v\_kind : v\_kind;
~ v\_ty   : ty;
\}

type expr =
| Pconst    of wsize \(\times\;\setZ\)
| Parr\_init of wsize \(\times\;\setN\)
| Pvar      of var
| Pget      of var \(\times\) expr
| Psub      of \(\setN\;\times\) var \(\times\) expr
\end{minipage}\hfill\vline\hfill\begin{minipage}{0.5\textwidth}
type lval =
| Lnone
| Lvar  of var
| Laset of var \(\times\) expr
| Lasub of \(\setN\;\times\) var \(\times\) expr

type instr =
| Cassgn of lval \(\times\) expr
| Ccall  of lval list \(\times\) uid \(\times\) expr list

type call\_conv =
| Export
| Subroutine

type func = \{
~ f\_name : uid;
~ f\_args : var list;
~ f\_body : instr list;
~ f\_ret  : var list;
~ f\_cc   : call\_conv;
\}

type prog = func list
\end{minipage}\normalfont%
\caption{Stripped and simplified version of Jasmin programs' AST type.}\label{fig:types}
\end{figure}


\section{Array expansion compiler pass}\label{sec:syntax}

\subsection{Introduction to relevant parts of Jasmin syntax}

In this section, the relevant parts of Jasmin programs' AST type is described.
The irrelevant parts, if mentioned, will not be expanded upon. Most of the types
are described in Figure \ref{fig:types} using a mix of a mathematical and
OCaml-like syntax.

A Jasmin progam is a list of functions. A function is made of its name, a unique
identifier; its parameters, a list of variables; its body, a list of
instructions; its returned variables, a list of variables; and its kind. The
kind of a function can be ``exported to the outside'', in which case the
function must follow some calling convention, or internal (subroutine).

\smallskip

An instruction can be an assignment or a function call. Other instructions
include assembly instructions and control flow instructions, however they are
not relevant to our work.

An assignment is made of an lvalue and an expression. A function call is
made of the called function identifier, a list of lvalues matching the
function returned variables list on length, and a list of expressions matching
the function parameters list on length.

Function arguments and return variables, lvalues and expressions refer to
variables. A variable is made of an identifier, a type and a kind. In the AST a
variable doesn't have to be declared in order to be used. Across a function
body, parameters and returned variables, all variables featuring the same
identifier and type are considered to be the same. A variable kind is a
constraint on where the compiler can store the value, it is either on the stack
or in a register.

\smallskip

An lvalue indicates where to store a value, it is either a variable
(\texttt{Lvar} constructor), a cell of an array variable (\texttt{Laset}), a
slice of an array variable (\texttt{Lasub}) or nowhere (\texttt{Lnone})
indicating the value will not be used and can be discarded as soon as possible.

An expression can be a constant word (\texttt{Pconst}), an empty array with no
initialised cell (\texttt{Parr\_init}), a variable (\texttt{Pvar}), a cell of an
array variable (\texttt{Pget}) or a slice of an array variable (\texttt{Psub}).
Operations like the addition of two expressions are also expressions.

Single array cells are indexed by expressions. Slice are contiguous sections of
arrays. The section is described using its length and offset from the begining
of the array. The length is a non-negative number and the offset an expression.

\smallskip

Arrays and words are two Jasmin types. A word is two to a non-negative power,
representing the size of the word in bytes. An array is a partial map from
non-negative numbers to words of some size.

\begin{figure}[t]
\obeylines\obeyspaces\ttfamily%
let rec expand\_e m = function
~ | Pvar x \(\mapsto\) if expand\_var m x is Some vx
~   then Ok (inr [:: Pvar v | v \(\gets\) vx ]) else Ok (inl (Pvar x))
~ | Pget x e \(\mapsto\) if expand\_var m x is Some vx then
~     if e is Pconst \_ z then if get vx z is Some v then Ok (inl (Pvar v))
~     else Error else Error
~   else Let inl e = expand\_e m e in Ok (inl (Pget x e))
~ | Psub len x e \(\mapsto\) if expand\_var m x is Some vx then
~     if e is Pconst \_ z then Ok (inr [:: Pvar v | v \(\gets\) take len (drop z vx) ])
~     else Error
~   else Let inl e = expand\_e m e in Ok (inl (Psub ws len x e))
~ | e \(\mapsto\) Ok (inl e)

let expand\_lv m = function
~ | Lvar x \(\mapsto\) if expand\_var m x is Some vx
~   then Ok (inr [:: Lvar v | v \(\gets\) vx ]) else Ok (inl (Lvar x))
~ | Laset x e \(\mapsto\) if expand\_var m x is Some vx then
~     if e is Pconst \_ z then if get vx z is Some v then Ok (inl (Lvar v))
~     else Error else Error
~   else Let inl e = expand\_e m e in Ok (inl (Laset x e))
~ | Lasub len x e \(\mapsto\) if expand\_var m x is Some vx then
~     if e is Pconst \_ z then Ok (inr [:: Lvar v | v \(\gets\) take len (drop z vx) ])
~     else Error
~   else Let inl e = expand\_e m e in Ok (inl (Lasub len x e))
~ | x \(\mapsto\) Ok (inl x)
\normalfont%
\caption{Expansion of expressions and lvalues.}\label{fig:fullexp}
\end{figure}

\subsection{Array expansion in Jasmin compiler}

A variable of type array and kind register means an array of registers, but in
most common processors on which cryptographic primitives are run there is no
hardware support for indexing registers with a register. Therefore we need to
eliminate them in some compilation pass; this is what the array expansion pass
does in the Jasmin compiler. Other kind of arrays are not expanded --even though
we could try to expand them-- because there is hardware support for indexing
memory with a register and so we consider users know what they want.

In order to expand an array we have to replace all store and load operations
performed on this array and modify function signature it appears in. The load
and store operations (resp. in expressions and lvalues) come in three kinds:
single cell indexing, array slicing and whole array access.

\smallskip

Assume we have a map associating each array to the list of variables it is
expanded to (called \texttt{m} in the following figures; as all array length are
known in Jasmin it is just a matter of generating enough fresh names).

\medskip

\begin{figure}[t]
\obeylines\obeyspaces\ttfamily%
let expand\_e\_n m n = function
~ | Pvar x       \(\mapsto\) if v\_ty x is Arr \_ \_
~   then Ok [:: Pget v (Pconst U64 i) | 0 \(\leq\) i \(<\) n ] else Error
~ | Psub len x e \(\mapsto\) Ok [:: Pget v (Padd e (Pconst U64 i)) | 0 \(\leq\) i \(<\) n ]
~ | \_            \(\mapsto\) Error
 
let expand\_lv\_n m n = function
~ | Lnone         \(\mapsto\) Ok [:: Lnone | 0 \(\leq\) i \(<\) n ]
~ | Lvar x        \(\mapsto\) if v\_ty x is Arr \_ \_
~   then Ok [:: Laset x (Pconst U64 i) | 0 \(\leq\) i \(<\) n ] else Error
~ | Lasub len x e \(\mapsto\) Ok [:: Laset x (Padd e (Pconst U64 i)) | 0 \(\leq\) i \(<\) n ]
~ | \_             \(\mapsto\) Error
\normalfont%
\caption{Expansion for unrolling expressions and lvalues.}\label{fig:unrexp}
\end{figure}

The array expansion for expressions and lvalues are depicted in Figure
\ref{fig:fullexp} using a pseudo OCaml-like syntax for convenience.
In this figure and all the following, \texttt{Let} \textit{pat} \texttt{=}
\textit{e1} \texttt{in} \texttt{e2} is a shorthand for matching the result of
\textit{e1} against \texttt{Ok} \textit{pat} and \texttt{Error}; in the former
case the relevant variables of \textit{pat} are bound and \textit{e2} is
evaluated, in the latter case the expression evaluates to \texttt{Error}.

The figure doesn't describe actual implemented code because expansion of
variables and subarrays for assignments is performed in a previous compilation
pass\footnote{Prior to our work, single cell access for lvalues and expression
in assignments was implmented in the array expansion pass and variable expansion
was implemented in a previous compilation pass}, but they could all be
implemented in one compiler pass as the code required to expand expressions and
lvalues for function calls and assignment only differs slightly as will be shown
further in this section.

\smallskip

Single cell indexing (\texttt{Pget} and \texttt{Laget}) can be expanded to a
single variable (\texttt{Pvar} and \texttt{Lvar}) when the index is a constant.
Otherwise it is not supported. When the index is a for loop variable in the
original program, the loop unrolling and constant propagation passes which come
before the array expansion pass have already transformed the indexes into
constants, so it also falls under the constant case.

Array slices (\texttt{Psub} and \texttt{Lasub}) are expanded by acceding to
the relevant slice of variables replacing the array provided the offset
parameter is a constant. The operation performed on this slice will then require
an unrolling.

Finally whole array accesses (\texttt{Pvar} and \texttt{Lvar}) are equivalent to
slices of arrays with no offset and same length as the base array on which they
are performed. They are replaced by all the variables the array is to be
replaced with.

\smallskip

In order to perform the unrolling we need to know whether the result of the
expansion of an expression or lvalue changed its type (\texttt{inr} cases) or
not (\texttt{inl} cases). For instace expanding a single cell access doesn't
change type (\texttt{Pget} or \texttt{Laget} are typed as word, and the variable
they are expanded to is also a word of the same size) even though an array
expansion is performed, whereas expanding a whole array does (array to several
words). We need this information because any instruction featuring expressions
or lvalues usually features the other and an array may have been expanded in one
but not in the other (because it is a register array in one side and not the
other), and only looking at the length of the expanded expressions and lvalues
isn't sufficient as arrays can be of length one.

For instance when expanding an assignment between single cell arrays, the
expanded expression and lvalue will be of length 1, which doesn't indicate
whether any array was expanded into a single word variable.

The expansion for unrolling expressions and lvalues (no array expansion
performed) is illustrated in Figure \ref{fig:unrexp} with the functions
\texttt{expand\_e\_n} and \texttt{expand\_lv\_n} to expand expressions and
lvalues when the other side expansion is of length \texttt{n}.

The expansions performed for unrolling array variables and slices consist in
accessing the relevant cells. The only other case where an unrolling makes sense
is the nowhere (\texttt{Lnone}) lvalue, which is the only one requiring a length
to know how many times it should be duplicated.

\medskip

\begin{figure}[t]
\obeylines\obeyspaces\ttfamily%
let expand\_fvar m v =
~ if expand\_var m v is Some vx then (Some (size vx), vx) else (None, [:: v ])

let expand\_fsig m fd =
~ let \{| f\_cc; f\_name; f\_args; f\_body; f\_ret |\} = fd in
~ if f\_cc = Export then (fd, None) else
~ let (args, expargs) = split [:: expand\_fvar m v | v \(\gets\) f\_args ] in
~ let (ret,  expret)  = split [:: expand\_fvar m v | v \(\gets\) f\_ret  ] in
~ (\{| f\_cc; f\_name; f\_args := flatten args; f\_body; f\_ret := flatten ret |\},
~   Some (expargs, expret))

let expand\_arg m ex e =
~ match ex, expand\_e m e with
~ | Some \_, Ok (inr es)  \(\mapsto\) Ok es
~ | Some n, Ok (inl e)   \(\mapsto\) expand\_e\_n m n e
~ | None  , Ok (inl e)   \(\mapsto\) Ok [:: e ]
~ | \_                    \(\mapsto\) Error

let expand\_ret m ex x =
~ match ex, expand\_lv m x with
~ | Some \_, Ok (inr xs)   \(\mapsto\) Ok xs
~ | Some n, Ok (inl x)    \(\mapsto\) expand\_lv\_n m n x
~ | None  , Ok (inl x)    \(\mapsto\) Ok [:: x ]
~ | \_                     \(\mapsto\) Error
\normalfont%
\caption{Expansion of signatures, argument expressions and returned values lvalues.}\label{fig:sigexp}
\end{figure}

Array expansion for function signatures and argument expressions and
returned values lvalues in function calls are depicted in Figure
\ref{fig:sigexp}.

The expansion of an array in function signature is simply a matter of replacing
it with the variables it is expanded to. A slight subtelty however is that like
expansion for expressions and lvalues we need to know which variables were
expanded in order to expand (sometimes without actual array expansion as
illustrated in \texttt{expand\_arg} and \texttt{expand\_ret}) arguments and
return assignments accordingly.

\begin{figure}[t]
\obeylines\obeyspaces\ttfamily%
let rec expand\_i fexpd m = function
~ | Cassgn x e \(\mapsto\)
~   match expand\_lv m x, expand\_e m e    with
~   | Ok (inl x)       , Ok (inl e)      \(\mapsto\) Ok [:: Cassgn x e ]
~   | Ok (inr xs)      , Ok (inr es)     \(\mapsto\) [:: Cassgn x e | x \(\gets\) xs, e \(\gets\) es ]
~   | Ok (inl x)       , Ok (inr es)     \(\mapsto\)
~     Let xs = expand\_lv\_n m x (size es) in [:: Cassgn x e | x \(\gets\) xs, e \(\gets\) es ]
~   | Ok (inl xs)      , Ok (inr e)      \(\mapsto\)
~     Let es = expand\_e\_n  m x (size xs) in [:: Cassgn x e | x \(\gets\) xs, e \(\gets\) es ]
~   | \_                                  \(\mapsto\) Error
~ | Ccall xs fn es \(\mapsto\)
~   if get fexpd fn is Some (expargs, expret) then
~     Let xs = [:: expand\_ret m xp x | xp \(\gets\) expargs, x \(\gets\) xs ] in
~     Let es = [:: expand\_arg m xp e | xp \(\gets\) expret , e \(\gets\) es ] in
~     Ok [:: Ccall (flatten xs) fn (flatten es) ]
~   else
~     Let xs = [:: Let inl x' = expand\_lv x in x' | x \(\gets\) xs ] in
~     Let es = [:: Let inl e' = expand\_e e  in e' | e \(\gets\) es ] in
~     Ok [:: Ccall xs fn es ]
\normalfont%
\caption{Expansion of instructions.}\label{fig:insexp}
\end{figure}

\smallskip

The expansion in function signature is only wanted when the function is not
exported to other programs because we do not want to make such changes unbeknown
to the user. Two solution satisfy this constraint\footnote{If register array
were allowed in exported functions by the typechecker}: warning the user or
failing to compile. As the global correction theorem only mentions exported
functions it would have to be complexified in order to reflect signature
expansion if we chose the former solution so we chose the latter.

\medskip

Expansion of instructions is illustrated in Figure \ref{fig:insexp}. As
described previously assignment are expanded by expanding each side of the
assignment and unrolling if necessary and function calls are expanded using
informating gathered while expanding the signature, which means signature
expansion needs to be performed first on every function before expanding the
body of any function.

\section{Semantics aspect of array expansion}\label{sec:semantics}

\subsection{Relevant parts of Jasmin semantics}\label{ssec:jassem}

\begin{figure}[t]
\obeylines\obeyspaces\ttfamily%
type value =
| Varr s n of array s n
| Vword s  of word  s

type sem\_t = function
| Arr s n \(\mapsto\) array s n
| Word s  \(\mapsto\) word  s

let type\_of\_val = function
~ | Vword s  \_ \(\mapsto\) Word s
~ | Varr s n \_ \(\mapsto\) Arr s n

let to\_arr  n v = if v is Varr s n t then Ok t else Error

let to\_word s v = if v is Vword s  w then Ok w else Error

let to\_val : sem\_t t \(\to\) value = function
~ | Arr s n \(\mapsto\) Varr s n
~ | Word s  \(\mapsto\) Vword s
\normalfont%
\caption{Simplified value types and related utilitary functions.}\label{fig:oldval}
\end{figure}

In this section, the relevant parts of Jasmin semantics are described. None of
it is part of our contribution but it is necessary in order to understand
the main properties we have to prove for the compiler correctness theorem.

Jasmin values can be words and partially defined arrays of words. Variable
values are stored in contexts using the dependent type \texttt{sem\_t} as
defined in Figure \ref{fig:oldval} in order to keep the invariant
\texttt{m.[x] = Ok v\(\implies\)type\_of\_val v = v\_ty x}.

\medskip

\begin{figure}[p] % TODO LATER: compare buty against sem rules
\obeylines\obeyspaces\ttfamily%
\begin{minipage}{0.52\textwidth}
let rec sem\_pexpr vm = function
~ | Pconst s z \(\mapsto\) Ok (Vword s (wrepr s z))
~ | Parr\_init s n \(\mapsto\)
~   Ok (Varr s n (array\_empty s n))
~ | Pvar v \(\mapsto\) vm.[v]
~ | Pget x e \(\mapsto\)
~   Let Varr s n t = vm.[x] in
~   Let vi = sem\_pexpr vm e in
~   Let i = to\_int vi in
~   Let w = array\_get t i in
~   Ok (Vword s w)
~ | Psub len x e \(\mapsto\)
~   Let Varr s n t = vm.[x] in
~   Let vi = sem\_pexpr vm e in
~   Let i = to\_int vi in
~   Let t' = array\_get\_sub len t i in
~   Ok (Varr s len t')
\end{minipage}\hfill\vline\hfill\begin{minipage}{0.45\textwidth}
let write\_lval l v vm =
~ match l with
~ | Lnone \(\mapsto\) Ok s
~ | Lvar x \(\mapsto\) vm.[x \(\gets\) v]
~ | Laset x e \(\mapsto\)
~   Let Varr s n t = vm.[x] in
~   Let vi = sem\_pexpr vm e in
~   Let i = to\_int vi in
~   Let w = to\_word s v in
~   Let t = set t i w in
~   vm.[x \(\gets\) to\_val (Arr s n) t]
~ | Lasub len x e \(\mapsto\)
~   Let Varr s n t = vm.[x] in
~   Let vi = sem\_pexpr vm e in
~   Let i = to\_int vi in
~   Let t' = to\_arr s len v in 
~   Let t = set\_sub n len t i t' in
~   vm.[x \(\gets\) to\_val (Arr s len) t]
\end{minipage}\normalfont%
\caption{Left: function evaluating expression e in context vm;}\vspace{-0.8\baselineskip}\center
Right: function modifying context s with value v according to lvalue l.\label{fig:semelv}
\end{figure}

The semantics of expressions and lvalues are illustrated in Figure
\ref{fig:semelv}.

Semantics of expressions is a partial function (\texttt{sem\_pexpr}) taking a
context and an expression and returning a value. The semantics of a constant is
the constant word with the same value. The semantics of an empty array is the
array value where no indices are defined. The semantics of a variable is the
value associated to that variable in the context. The semantics of indexing an
array variable (\texttt{Pget x e}) is the word at the index given by the
semantics of the index expression (\texttt{e}) of the array associated to the
variable (\texttt{x}) in the context. The semantics of an array slice
\texttt{Psub len x e} is the array value obtained by slicing the array
associated to the variable \texttt{x} in the context, slice of length
\texttt{len} and starting at the offset given by the semantics of the offset
expression \texttt{e}.

Semantics of lvalues is a partial function (\texttt{write\_lval}) taking an
lvalue, a value and a context, and returning a context.
The semantics of the nowhere (\texttt{Lnone}) is the unchanged context. The
semantics of a variable (\texttt{Lvar x}) is the context where the variable
(\texttt{x}) is associated to the given value. The semantics of indexing an
array variable (\texttt{Laset x e}) is the context where the array associated
to the variable (\texttt{x}) has its cell given by the semantics of the index
expression (\texttt{e}) associated to the given value. The semantics of slicing
an array variable (\texttt{Lasub len x e}) is the context where the array
variable (\texttt{x}) is associated to an array which \texttt{len} cells
starting from the index given by the semantics of the offset expression
(\texttt{e}) are associated to the \texttt{len} cells from the array value
given.

\medskip

\begin{figure}[p]
\ttfamily
sem\_call P :
\begin{center}
\begin{tabular}{c}
get\_func P fn = Some f \\
fold2 (\(\lambda\)x v, vm.[x \(\gets\) v]) (f\_args f) vargs empty = Ok vm1 \\
sem P vm1 (f\_body f) vm2 \\
{}[:: vm2.[x] | x \(\gets\) f\_ret f ] = Ok vres
\\\hline
sem\_call P fn vargs vres
\end{tabular}
\end{center}

sem P :
\begin{center}
\begin{tabular}{c}
\\\hline
sem P vm [::] vm
\end{tabular} \textrm{\textsc{Eskip}}\quad
\begin{tabular}{c}
sem\_i P vm1 i vm2\quad
sem P vm2 l vm3
\\\hline
sem P vm1 (i::l) vm3
\end{tabular} \textrm{\textsc{Eseq}}
\end{center}

sem\_i P :
\begin{center}
\begin{tabular}{c}
sem\_pexpr vm1 e = Ok v\quad
write\_lval x v vm1 = Ok vm2
\\\hline
sem\_i P vm1 (Cassgn x e) vm2
\end{tabular} \textrm{\textsc{Eassgn}}

~

\begin{tabular}{c}
[:: sem\_pexpr vm1 e | e \(\gets\) es ] = Ok vargs \\
sem\_call P f vargs vres \\
fold2 write\_lval xs vres vm1 = Ok vm2
\\\hline
sem\_i P vm1 (Ccall xs f es) vm2
\end{tabular} \raisebox{-\baselineskip}{\textrm{\textsc{Ecall}}}
\end{center}
\normalfont%
\caption{Simplified inductive big step semantics rules.}\label{fig:semall}
\end{figure}

The semantics of Jasmin functions and instructions in a program \texttt{P} are
defined as the big step inductive predicates \texttt{sem\_call} and
\texttt{sem\_i} as depicted in Figure \ref{fig:semall}. The former relates a
function and its arguments to the returned values, the latter relates a context
and an instruction to the context after executing the instruction. The semantics
for instructions is naturally extanded to a sequence of instructions in
predicate \texttt{sem} by chaining the contexts (\textsc{Eseq}) with the empty
sequence being the identity on contexts (\textsc{Eskip}).

\smallskip

The semantics of a function call is made of only one rule with four premisses:
the function identifier is bound to a function (\texttt{f}) of the program,
binding this function arguments to the parameters in an empty context gives a
valid context (\texttt{vm1}), the semantics of sequences of instructions
(\texttt{sem}) relates this valid context and the function body to another
context (\texttt{vm2}) and in this latter context the values associated to the
return variables of the function correspond to the return values of the jugement.

\smallskip

The semantics of an instruction is made of two rules. The rule for assignment
(\textsc{Eassgn}) gives the semantics of assignment instructions
(\texttt{Cassgn x e}), it has two premisses: the semantics of the assignment
expression (\texttt{e}) in the initial context (\texttt{vm1}) is a value
(\texttt{v}) and the semantics of the assignment lvalue (\texttt{x}) on the
initial context (\texttt{vm1}) and that value (\texttt{v}) correspond to the
final context (\texttt{vm2}) of the jugement. The rule for function calls
(\textsc{Ecall}) gives the semantics of the call instruction
(\texttt{Ccall xs f es}), it has three premisses: the semantics of the argument
expressions (\texttt{es}) in the initial context (\texttt{vm1}) are values
(\texttt{vargs}), the semantics of function call relates the function
(\texttt{f}) and these values (\texttt{vargs}) to other values (\texttt{vres})
and the semantics of the lvalues (\texttt{xs}) chained on contexts, starting
from the initial context (\texttt{vm1}), on the latter values (\texttt{vres})
correspond to the final context (\texttt{vm2}) of the jugement.


\subsection{Semantic implications of array expansion}

In this section the semantics changes of all the cases for array expansion are
described and their implications are discussed. Semantic preservation lemmas are
illustrated in Figure \ref{fig:semlem} using a mix of Coq\footnote{Most of
developpement presented previously is in Coq, but for proofs there is no simpler
equivalent}-like and mathematical syntax for lemmas.

\smallskip

\begin{figure}[p]
\obeylines\obeyspaces\ttfamily%
let eval\_array s t i = Let w = array\_get t i in Vword s w

let expand\_v e v = if e then
~   if v is Arr s n t then [:: eval\_array s t i | 0 \(\leq\) i \(<\) n ] else Error
~ else Ok [:: v ]

Definition eq\_alloc m vm1 vm2 =
~ \(\forall\)x, match get m x with
~ | Some vx \(\mapsto\) \(\forall\)i vi, get vx i = Some vi \(\Longrightarrow\)
~   (Let Varr s n t = vm1.[x] in eval\_array t i) = vm2.[vi]
~ | None \(\mapsto\) vm1.[x] = vm2.[x]
~ end.

Lemma expand\_ePl m vm1 vm2 :
~ eq\_alloc m vm1 vm2 \(\Longrightarrow\)
~ \(\forall\)e1 e2, expand\_e m e1 = Ok (inl e2) \(\Longrightarrow\)
~ \(\forall\)v, sem\_pexpr vm1 e1 = Ok v \(\Longrightarrow\) sem\_pexpr vm2 e2 = Ok v.

Lemma expand\_ePr m vm1 vm2 :
~ eq\_alloc m vm1 vm2 \(\Longrightarrow\)
~ \(\forall\)e1 es2, expand\_e m e1 = Ok (inr es2) \(\Longrightarrow\)
~ \(\forall\)v, sem\_pexpr vm1 e1 = Ok v \(\Longrightarrow\)
~ \(\exists\)vs, expand\_v true v = Ok vs \(\wedge\) [:: sem\_pexpr vm2 e | e \(\gets\) es2 ] = Ok vs.

Lemma expand\_lvPl m vm1 vm2 :
~ eq\_alloc m vm1 vm2 \(\Longrightarrow\)
~ \(\forall\)x1 x2, expand\_lv m x1 = Ok (inl x2) \(\Longrightarrow\)
~ \(\forall\)v vm1', write\_lval x1 v vm1 = Ok vm1' \(\Longrightarrow\)
~ \(\exists\)vm2', write\_lval x2 v vm2 = Ok vm2' \(\wedge\) eq\_alloc m vm1' vm2'.

Lemma expand\_lvPr m vm1 vm2 :
~ eq\_alloc m vm1 vm2 \(\Longrightarrow\)
~ \(\forall\)x1 xs2, expand\_lv m x1 = Ok (inr xs2) \(\Longrightarrow\)
~ \(\forall\)v vm1', write\_lval x1 v vm1 = Ok vm1' \(\Longrightarrow\)
~ \(\exists\)vs, expand\_v true v = Ok vs \(\wedge\)
~   \(\exists\)vm2', fold2 write\_lval xs2 vs vm2 = Ok vm2' \(\wedge\) eq\_alloc m vm1' vm2'.

Lemma expand\_funcP P1 P2 fexpd fn vargs vret =
~ sem\_call P1 fn vargs vret \(\Longrightarrow\)
~ \(\forall\)xarg xret, get fexpd fn = Some (xarg, xret) \(\Longrightarrow\)
~ \(\forall\)vargs', [:: expand\_v (is\_some x) v | x \(\gets\) xarg, v \(\gets\) vargs ] = Ok vargs' \(\Longrightarrow\)
~ \(\exists\)vret',  [:: expand\_v (is\_some x) v | x \(\gets\) xret, v \(\gets\) vret  ] = Ok vret' \(\wedge\)
~   sem\_call P2 fn (flatten vargs') (flatten vres').

Lemma expand\_iP P1 P2 fexpd m vm1 vm2 =
~ eq\_alloc m vm1 vm2 \(\Longrightarrow\)
~ \(\forall\)i1 vm1', sem\_i P1 vm1 i1 vm1' \(\Longrightarrow\)
~ \(\forall\)is2, expand\_i fexpd m i1 = Ok is2 \(\Longrightarrow\)
~ \(\exists\)vm2', sem P2 vm2 is2 vm2' \(\wedge\) eq\_alloc m vm1' vm2'.
\normalfont%
\caption{Lemmas relating the semantics before and after array expansion.}\label{fig:semlem}
\end{figure}

As values and contexts are at the core of semantics, the semantics preservation
theorem has to express relations between them and their expanded version. The
relation between a value and its expanded version --a list of values-- is
\texttt{expand\_v}, it takes a boolean argument to express whether the value
should be expanded or not. If the value is expanded then it is an array and its
expansion is the list of its cells. The relation between a context and its
expanded version is \texttt{eq\_alloc}, it takes the same information for
expanding variables as the AST expansion. The relation holds if all expanded
arrays have their cell equal to the variable to which it is expanded in the
expanded context and all the other variables are equal in both contexts.

\medskip

Assume we have a program \texttt{P1}, its expansion \texttt{P2} and the
expansion information \texttt{fexpd} gathered when expanding the function
signatures (\texttt{expand\_fsig} in Figure \ref{fig:sigexp}).

\smallskip

The semantics of a function call after array expansion takes an expanded list
of values as arguments, executes the expaded body on these variables and returns
the values of the expanded returned variables. Hence the semantics preservation
lemma for expanded functions (\texttt{expand\_funcP}) assumes the original
semantics and expansion of arguments and requires the semantics of the expanded
functions on expanded arguments to correspond to the expanded list of return
values.

The semantics of an instruction after array expansion takes an expanded context,
executes instruction on it and returns an expanded context. Hence the semantics
preservation lemma for expanded instructions (\texttt{expand\_iP}) assumes the
original semantics relates a context \texttt{vm1} to a context \texttt{vm1'},
the first context \texttt{vm1} is related with the expansion relation to an
initial context \texttt{vm2} for executing the expanded instruction and requires
the context \texttt{vm1'} to be related with the expansion relation to a context
\texttt{vm2'} and the semantics of the expanded instructions relates the expanded
initial context \texttt{vm1'} to the expanded final context \texttt{vm2'}.

\smallskip

After array expansion, an expression is either a single expression (\texttt{inl}
case) or it becomes a list of expressions (\texttt{inr} case). Their semantics
in the expanded context is a list of values. Hence the semantics preservation
lemmas for expanded expressions (\texttt{expand\_ePl} and \texttt{expand\_ePr})
assume the original semantics in a context \texttt{vm1} is a value \texttt{v},'
the orgiginal context \texttt{vm1} is related with the expansion relation to a
context \texttt{vm2} and requires to prove that the expanded expressions
semantics in the expanded context \texttt{vm2} corresponds to the expansion of
value \texttt{v}, or when the expansion is a single expression that this
expression semantics is the same as the original expression semantics.

After array expansion, a lvalue is either a single lvalue (\texttt{inl} case) or
it becomes a list of lvalues (\texttt{inr} case). Their semantics starting from
an expanded context on an expanded value is an expanded context, and in the case
the lvalue is expanded to a list of lvalues their semantics is chained on the
contexts. Hence the semantics preservation lemmas for expanded lvalues
(\texttt{expand\_lvPl} and \texttt{expand\_lvPr}) assume the semantics of the
original lvalue on a value \texttt{v} relates a context \texttt{vm1} to a
context \texttt{vm1'}, the initial context \texttt{vm1} is related using the
expansion relation to a context \texttt{vm2} and requires to prove the final
context \texttt{vm1'} is in relation to an expanded context \texttt{vm2'}, using
expansion relation, corresponding to the final context related to the context
\texttt{vm2} by the semantics of the expanded lvalue on the expansion of value
\texttt{v}.

\medskip

With these lemmas proved, proving the semantics preservation theorem is just a
matter of proving the exported functions are unmodified and using the semantic
preservation lemma for function. However this lemma is not provable: Calling a
function requires to build a context where each parameter receives a value, but
the array semantics is partial arrays so some cells are not defined and
accessing their value fails, so programs where some functions were called on
partial arrays wouldn't have a semantics after the transformation. It is not an
issue when comparing contexts as the objects compared in \texttt{eq\_alloc} are
either \texttt{Ok} of some value or \texttt{Error} which account for the case
where the access to an array cell or a variable fails.


\section{Pervasive changes to Jasmin semantics}\label{sec:perch}

\begin{figure}[t]
\obeylines\obeyspaces\ttfamily%
type value =
| Varr s n of array s n
| Vword s  of word  s
| Vundef

let type\_of\_val = function
~ | Vundef     \(\mapsto\) Word 0
~ | Vword s  \_ \(\mapsto\) Word s
~ | Varr s n \_ \(\mapsto\) Arr s n
\normalfont%
\caption{New value type.}\label{fig:newval}
\end{figure}

In this section the changes made to be able to prove the semantics preservation
theorem are described and motivated. This part takes the most time of all as the
changes made are at the core of the semantics, thus affecting every compilation
pass, requiring the proof of every compilation pass to be fixed. Because of the
change affecting all compilation passes, less simplifications than in the
previous sections will be made, exposing some details previously hidden.

\smallskip

The issue raised by array expansion preventing us from proving semantics
preservation is caused by the semantics of arrays which are partial arrays, so
the fix requires to modify the semantics\footnote{or give up some parts of array
expansion}. It can be solved either by eliminating parially initialised arrays,
requiring arrays to be completely initialised, tracking the initialised cells
(which would require some kind of reduction of expressiveness as otherwise it
isn’t computable), store optional words instead of words in the context or allow
values to be undefined. We chose the last solution as undefined values were
already in the value type for booleans (which were hidden in previous sections
because of irrelevance).

The new value type described in Figure \ref{fig:newval} features the undefined
word value (arrays are always defined even though some cells may not be) and
context have been modified to store values of the new \texttt{value} type
instead of the \texttt{sem\_t} dependent type.

\smallskip

There are two main motivations for replacing the dependent type in the contexts.
The first is to extend undefined values to the word type, which could be done
several ways without removing dependent types, for instance by using an option
word type instead of the word type
(\(\texttt{sem\_t} : \texttt{Word s} \mapsto \texttt{option (word s)}\)).
The second motivation is that eliminating dependent types makes invariants
explicit and tend to simplify the proofs.

Proofs on dependent type tend to be tiresome because every rewriting need to
ensure types are not violated, which in the case of dependent type means
ensuring the parameters given to the dependent type are rewritten uniformly in
the goal and assumptions. In particular in proofs using variables we sometimes
weren't able to rewrite the \texttt{v\_ty} field without extra work. This need
is eliminated by using a non-dependant value type since all values are of the
\texttt{value} type instead of being of one of \texttt{word}, \texttt{array} or
the other types.

Removing dependent types where unnecessary and favoring decidable alternatives
is also a design choice made by SSReflect, a widespread Coq proof library used
in Jasmin. The drawbacks of not using dependent types are that invariants need
to be stated and proved manually, which is actually not that much of a drawback
since the invariants become explicit and the proof are usually easy to do.

\medskip

\begin{figure}
\obeylines\obeyspaces\ttfamily%
type pword s = \{
~ pw\_size  : wsize;
~ pw\_word  : word pw\_size;
~ pw\_proof : pw\_size \(\leq\) s;
\}

type psem\_t = function
| Word s \(\mapsto\) pword s
| t      \(\mapsto\) sem\_t t
\normalfont%
\caption{\texttt{psem\_t} dependent type definition.}\label{fig:psemt}
\end{figure}

The old contexts used a dependent type to store the values, this dependent type
was either \texttt{sem\_t} or \texttt{psem\_t}. The latter described in Figure
\ref{fig:psemt} is used to store smaller words into seemingly bigger words for
a compilation pass where all registers and words become 64 bits as is available
in a x86 processor. The old invariant the context maintained using the dependent
types was that the type of the variable corresponded to the type of the value
stored, since the type of the value stored was the dependent type applied to the
type of the variable, which in turn gave the invariant presented in Section
\ref{ssec:jassem}
\texttt{vm.[x] = Ok v \(\Longrightarrow\) type\_of\_val v = v\_ty x} for free.

\begin{figure}[t]
\obeylines\obeyspaces\ttfamily%
let subtype t1 t2 =
~ match t1, t2 with
~ | Word s1, Word s2 \(\mapsto\) s1 \(\leq\) s2
~ | \_,       \_       \(\mapsto\) t1 = t2

let compat\_type t1 t2 =
~ match t1, t2 with
~ | Word \_, Word \_ \(\mapsto\) true
~ | \_,      \_      \(\mapsto\) t1 = t2

let is\_word t = if t is Word \_ then true else false
\normalfont%
\caption{Subtyping relation and type compatibility relation.}\label{fig:typerel}
\end{figure}

The new context store their value in the new \texttt{value} type. To account for
the two cases where we have words of the right size or of a smaller size
allowed, the contexts take a type relation as parameter, this type relation is
either equality or subtyping (\texttt{subtype}) only on words as defined in
Figure \ref{fig:typerel}. We need to manually ensure the previous context
invariant introduced in Section \ref{ssec:jassem} is still maintained.
Previously the access to a variable could fail if the variable is undefined and
a word for instance, now access to undefined word variable succeeds with the
undefined value so there is no fail case. Therefore the invariant becomes
\texttt{if vm.[x] = Vundef then is\_word (type\_of\_val vm.[x]) else (rel vm) (type\_of\_val vm.[x]) (v\_ty x)}
where \texttt{rel vm} is the type relation parameter of the context \texttt{vm}.

\medskip

\begin{figure}[t]
\obeylines\obeyspaces\ttfamily%
let word\_uincl s1 s2 (w1 : word s1) (w2 : word s2) =
~ s1 \(\leq\) s2 \(\wedge\) w1 = zero\_extend s1 w2

let array\_uincl s1 n1 s2 n2 (a1 : array s1 n1) (a2 : array s2 n2) =
~ n1 = n2 \(\wedge\;\bigwedge_{i = 0}^{n1 - 1}\) if array\_get a1 i is Ok w then array\_get a2 i = Ok w else false

let value\_uincl v1 v2 =
~ match v1, v2 with
~ | Varr s1 n1 t1, Varr s2 n2 t2 \(\mapsto\) array\_uincl s1 n1 s2 n2 t1 t2
~ | Vword s1 w1,   Vword s2 w2   \(\mapsto\) word\_uincl s1 s2 w1 w2
~ | Vundef,        Vword \_ \_     \(\mapsto\) true
~ | \_,             \_             \(\mapsto\) false

Lemma value\_uincl\_truncate ty x y x' :
~ value\_uincl x y \(\Longrightarrow\) truncate\_val ty x = Ok x' \(\Longrightarrow\)
~ \(\exists\)y', truncate\_val ty y = Ok y' \(\wedge\) value\_uincl x' y'.

Lemma set\_get vm vm' x v : vm.[x \(\gets\) v] = Ok vm' \(\Longrightarrow\)
~ vm'.[x] = if rel (type\_of\_val v) (v\_ty x) then v else truncate\_val (v\_ty x) v.
\normalfont%
\caption{Core lemma used to prove semantics preservation.}\label{fig:corlem}
\end{figure}

% TODO: from HERE

The semantics of instructions and function call presented in the previous section
was simplified. In particular because of the step where all stored words become
64 bits, they are truncated to the size they are supposed to be before being
assigned and after being retrieved from the context. The truncation function
has to be extended to undefined values. There are several ways to do this
extension, but the truncation function should still verify two core properties
to do most of the proofs of semantics preservation in all compilation passes:
\begin{itemize}
\item If assigning a value to a variable in a context succeeds, then the value
  associated to the variable is either the value assigned if the relation of the
  context is satisfied between the type of the value and the type of the
  variable or it is the truncation of the value assigned to the type of the
  variable (if the type of the value is bigger than the type of the variable,
  because it is the only other case where the assignment succeeds).
\item If a value is an extension (array more defined or bigger word of same
  value) of another then truncating them to the same type preserve this
  extension relation.
\end{itemize}
The first property is needed to be able to store smaller words into 64 bits word
variables and it forces the truncation to be the identity when the truncation
type is bigger than the value type.

\begin{figure}[t]
\obeylines\obeyspaces\ttfamily%
let truncate\_val ty = function
~ | Varr s n \_ \(\mapsto\) if ty = Arr s n then Ok v else Error
~ | Vword s w  \(\mapsto\) if ty is Word s'
~   then Ok (if truncate\_word s' w is Ok w' then Vword s' w' else v) else Error
~ | Vundef     \(\mapsto\) if ty is Arr \_ then Ok Vundef else Error
\normalfont%
\caption{Truncation of values to use with the new contexts.}
\end{figure}

If we chose the truncation to be the identity in the undefined case and the
truncation of a word to a bigger size (where previously it failed) then both
properties are verified.

\medskip

With these modifications we are confident the correctness proof should be
provable. At the writing of this article the correctness proof for the register
allocation pass succeeded and the proof for the register array expansion is in
progress.

\section{Conclusion}\label{sec:ccl}
% Présentation du travail restant dans le stage ainsi que potentiellement après
% le stage: FIXER LES PREUVES
% ouverture: array expansion is not limited to our compiler, you could implement
% your own version too and get handy array interface on variables

\section{Bibliography}
\bibliographystyle{plain}
\bibliography{biblio}

\end{document}
